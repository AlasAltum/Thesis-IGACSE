\documentclass{umemoria} 

\depto{Departamento de Ciencias de la Computación}
\author{Alonso Utreras Miranda}
\title{IGACSE: Un videojuego educativo para enseñar algoritmos relacionados a grafos a estudiantes de Ciencias de la Computación}

% incluir ambos comandos para una doble titulación
%  o quitar el comando que no aplica
\memoria{Ingeniero Civil en Computación}
\tesis{Magíster en Ciencias, mención Computación}
%\tesis{Doctor en ???} % incluir solo este comando para doctorados

% puede haber varios profesores guía seperados por coma;
% pero si es una memoria, solo puede haber un profesor guía
\guia{Iván Sipirán Mendoza} 

% puede haber varios profesores co-guía seperados por coma;
% pero si es una memoria, el profesor co-guía será el primer
% integrante de la comisión
%\coguia{Nombre Completo Co-Guía} % incluir en caso de co-guía de *tesis*

%\cotutela{Nombre Institución} % incluir en caso de cotutela
% TODO: Agregar comision
\comision{Juan Álvarez, Nelson Baloian , Federico Meza}

%\auspicio{Nombre Institución} % incluir en caso de recibir financiamiento

% tiene que ser el año en que se da el examen de título/grado (defensa)
%\anho{2021} % incluir solo para reemplazar el año actual

\begin{document}

\frontmatter
\maketitle

\begin{resumen}

Los videojuegos educativos se utilizan como herramientas destinadas a motivar y entretener durante el proceso de aprendizaje. Para respaldar estas afirmaciones, se han desarrollado marcos de trabajo y revisiones de literatura que analizan los enfoques utilizados en los estudios relacionados con los videojuegos educativos, como por ejemplo MEEGA+.

En este estudio se crea y pone a prueba un videojuego educativo en el campo de la computación, centrándose en el contenido de grafos. Para esto, se desarrolla IGACSE, un videojuego educativo diseñado para enseñar sobre grafos y los algoritmos Búsqueda en Anchura (Breadth First Search o BFS) y Búsqueda en Profundidad (Depth First Search o DFS). La elección de grafos se sustenta en que estas estructuras de datos tienen una representación visual directa, facilitando el diseño del videojuego.

Para la evaluación cuantitativa de IGACSE, se evalúan dos factores: la percepción del usuario y su aprendizaje. Para el primer aspecto, se utiliza como referencia metodológica el enfoque de MEEGA+, que permite medir la percepción general de los participantes respecto al juego a través de un valor numérico; para el segundo aspecto, se estableció una prueba académica para medir el aprendizaje adquirido en el juego. Como muestra se consideran dos grupos, el primer grupo (N=15) para la medición de percepción y aprendizaje; y un segundo grupo (N=15) para validar solo los resultados de percepción del videojuego.

Los resultados obtenidos sugieren que el juego facilitó el aprendizaje de los estudiantes, pero se identifican espacios de mejora a nivel de diseño de juego. Además, es necesario llevar a cabo un mayor número de pruebas empleando diferentes metodologías de estudio y muestras de tamaño más significativo, con el fin de otorgar una mayor validez científica a los resultados obtenidos. La percepción de los usuarios señaló posibles áreas de mejora en usabilidad, interfaz y ludificación. La arquitectura de software desarrollada permite su reutilización con otros algoritmos y estructuras de datos. Además, se proporcionan pautas para obtener resultados más precisos en futuras investigaciones relacionadas.


\end{resumen}


% \begin{abstract}

% Educational video games are acknowledged as motivating tools for teaching and entertaining students. In this project, IGACSE is introduced, an educational video game designed to impart concepts of graphs and BFS and DFS algorithms. The MEEGA+ methodology \cite{meegaplus} was employed to assess the quality of the application, involving two groups to play the game and respond to questionnaires. The first group underwent an academic test following the questionnaire. This group, unfamiliar with graph concepts, achieved a perfect score (n=15) after testing the video game and rated it with $\theta = 0.613$, which, according to MEEGA+ \cite{MeegaPlusManual}, corresponds to a good but not excellent game. The second group, which did not take the academic test, gave a score of $\theta = 0.616$. It is concluded that the game facilitated student learning about graphs and BFS and DFS algorithms. User perception indicated potential areas for improvement in usability, interface, and gamification. The software architecture developed allows for reuse with other algorithms and data structures. At the end, guidelines are provided for obtaining more precise results in future research.

% \end{abstract}

\begin{dedicatoria}
A mi abuelo, que siempre me apoyó en mis estudios. 
\end{dedicatoria}

\begin{thanks}
    Gracias a Noemí por acompañarme en todo este proceso, de inicio a fin. A mi familia por su apoyo incondicional en todos los niveles.

    A mí profesor guía y maestro, Iván Sipirán, quien siempre me recibió con una sonrisa, una conversación muy amena y dando excelentes consejos.

    A mis profesores por su ayuda, por ayudarme a crecer no solo como profesional y como estudiante, sino también como persona. No sería quien soy de no ser por ustedes. 

    Gracias a mis amistades, por su apoyo, pero también a quienes me pidieron apoyo, porque en la ayuda mutua es donde surgen las mejores ideas. Gracias Alfonso, Gabriel, Ricardo, Felipes, Jeremy, Nancy, Isa, Enri, Fernanda, Mario, Cisneros, Dani, Nico, Lucas, Mati, Martín, Bea, Tommy, Diego y Vale.

    Gracias a mis colegas, que me ayudaron sin esperar nada a cambio. Me dieron excelentes ideas, jamás se me hubieran ocurrido.

    Gracias a la Facultad por brindarme el espacio de trabajo, donde disfruté y aprendí tanto. Por permitirme trabajar de lunes a domingo. Por poder pasar la noche aquí y tener un ambiente de trabajo propicio.

\end{thanks}

\tableofcontents
% \listoffigures % opcional
% \listoftables % opcional

\mainmatter

\chapter{Introducción}

Esta investigación tiene como objetivo transformar las instrucciones de algoritmos de grafos en una representación visual e interactiva a través de un videojuego. Esta representación se destina a estudiantes de informática, con lo que se puede evaluar si el uso de herramientas interactivas con feedback visual mejora tanto el aprendizaje como la motivación.

El trabajo implica presentar a estudiantes de ciencias de la computación (CS) un videojuego que exhiba grafos. En este juego, el usuario debe ejecutar las instrucciones de los algoritmos de grafos con la asistencia de elementos interactivos de caracter visual y auditivo.

El objetivo de esta investigación de tesis es identificar posibles diferencias en los niveles de motivación y comprensión de los algoritmos relacionados a grafos entre los estudiantes. Esto se medirá mediante una prueba que evaluará su conocimiento de estos procesos. Los algoritmos a enseñar y evaluar son BFS y DFS.

El público objetivo son estudiantes de primer año de ciencias de la computación, aunque también es aplicable a estudiantes de ingeniería con conocimientos de programación. El requisito principal es que no hayan estudiado grafos previamente.

\section{Motivación}

La adopción de tecnologías digitales ha agilizado el acceso al conocimiento, permitiendo que estudiantes previamente excluidos ahora puedan acceder a la educación. Sin embargo, la educación en línea presenta desafíos propios \cite{UN2023ImpactDigitalTechnologies}. En este contexto, es crucial buscar metodologías que aceleren el aprendizaje con tecnologías adaptables, escalables y eficientes.

Se postula popularmente que la capacidad de atención de forma prolongada ha disminuido en las nuevas generaciones. Hay estudios que contradicen estas afirmaciones \cite{The_Role_of_Attention_Learning_Digital_Age}, indicando que las habilidades cognitivas de los estudiantes han cambiado, pero no necesariamente empeorado. Existen términos como ``doomsters'' y ``boosters'' \cite{Selwyn2014LookingF}, para descibrir la polarización entre estas miradas con respecto a la tecnología.

Existe consenso en que la tecnología y su portabilidad han incrementado notablemente la exposición de los estudiantes a distracciones \cite{Zimmerman2011HandbookOS, Wang2022ComprehensivelySummarizeDistractions}. El término "multitasking" se refiere al intento de llevar a cabo múltiples tareas simultáneamente. Aunque las personas tienden a creer que son capaces de realizar varias actividades de manera concurrente, diversos estudios han demostrado que esta práctica conlleva una disminución en la productividad y en el proceso de aprendizaje \cite{Domoff2019AddictivePU}. Con la generalización de los smartphones y la adopción de clases en línea, se ha observado un aumento significativo en la tendencia al multitasking durante las clases \cite{Wang2022ComprehensivelySummarizeDistractions}.

Una forma de distracción reconocida en la literatura es la interferencia motivacional, propuesta y explicada por Fries y Dietz \cite{Fries2007LearningMotivationalInterference}. Esta teoría postula que la motivación por los contenidos disminuye ante la presencia de estímulos más atractivos para la atención, tales como los smartphones. En este contexto, donde los estudiantes enfrentan la tentación de distraerse, resulta fundamental explorar estrategias destinadas a mantener su motivación y enfoque durante las clases.

Los videjuegos educativos destacan porque tienen potencial como una herramienta complementaria para la enseñanza, evaluación y entretemiento para los estudiantes. Entre sus beneficios se destaca la motivación. En \cite{Bisson1996FunInLEarningPedagogicalRole} se indica que para disfrutar una actividad, primero se debe permitir a la mente de un individuo percibir tal actividad como motivante. 

Yu et al. \cite{Yu2020TheEffectsOfEducationGames} llevan a cabo una revisión sistemática de la literatura acerca de los efectos de los videojuegos educativos en el aprendizaje de los estudiantes y su motivación. En relación con este último aspecto, el estudio señala que la incorporación de videojuegos educativos como recurso complementario impacta positivamente en la motivación y, además, mejora los logros académicos. Sin embargo, también destaca la existencia de investigaciones que contradicen esta afirmación, subrayando así la necesidad de llevar a cabo más estudios en esta área.

En el estudio mencionado, se asevera que el diseño del videojuego tiene una gran incidencia en el resultado final. Estos concluyen que las mecánicas, elementos visuales y narrativos tienen un efecto significativo, sugiriendo que los juegos educativos deberían implementar varios elementos de gamificación, como rankings, sistemas de recompensa, entre otros aspectos \cite{Yu2020TheEffectsOfEducationGames}.

Yu et al. \cite{Yu2020TheEffectsOfEducationGames} además mencionan una ventaja asociada a los videojuegos educativos, y es que pueden proveer servicios educacionales de alta calidad, flexible, portable y de bajo costo, incrementando las interacciones entre materiales de aprendizaje, estudiantes y profesores. Considerando lo anterior, se presenta una oportunidad para crear un videojuego educativo que enseñe algoritmos relacionados a grafos y analizar la percepción de sus usuarios.

En más del 50\% de los estudios citados anteriormente, se señala la falta de certeza con respecto a la percepción de los usuarios al jugar videojuegos educativos, concluyendo que se requiere profundizar la investigación. Por lo tanto, resulta pertinente emplear una prueba estandarizada que permita evaluar diversos diseños de videojuegos, abarcando distintas poblaciones objetivo, y cuyos resultados sean medidos conforme a un estándar uniforme. Con este propósito, se utilizó un formulario basado en el modelo MEEGA+ \cite{meegaplus} para evaluar la motivación de los estudiantes, y una prueba de conocimientos para medir su aprendizaje.


\section{Preguntas de Investigación}

\emph{Q1} ¿Puede un grupo de estudiantes que no tenga conocimiento previo sobre grafos, identificar y construir un recorrido BFS y DFS solo habiendo sido expuesto a un videojuego educativo sobre grafos?

\emph{Q2}: ¿Cómo perciben los estudiantes de ciencias de la computación sin conocimientos de grafos un videojuego educativo sobre grafos?


\section{Hipótesis}

\emph{H1}: Un videojuego que utilice conceptos relacionados a grafos puede enseñar sobre los algoritmos de BFS y DFS.

\emph{H2}: Un videojuego que enseñe grafos será percibido de manera positiva por los estudiantes que todavía no aprenden sobre esos contenidos.


\section{Objetivo General}

El objetivo principal de este trabajo es medir el aprendizaje y la motivación de los estudiantes de computación al utilizar un videojuego educativo para instruir en algoritmos vinculados a grafo.

\section{Objetivos Específicos}

\begin{itemize}

\item Concebir una aplicación interactiva que visualice grafos y permita seguir los pasos relacionados con algoritmos que operan en dichos grafos. Esta aplicación debe tener elementos característicos de los videojuegos educativos.

\item Idear, desarrollar e implementar mecánicas de juego y una arquitectura de programación transferibles a otros videojuegos que instruyan en materias relacionadas con la programación.

\item Llevar a cabo una evaluación estandarizada para medir la percepción de los estudiantes respecto al videojuego creado.

\end{itemize}


\section{Marco teórico}

Es importante comprender sobre las tres ideas que se explicarán a continuación para entender mejor el trabajo. Primero que todo, comprender cómo se han llevado a cabo los estudios de videojuegos educativos y por qué estos requieren mayor rigor científico. Luego, entender el rol de un motor de videojuegos y cómo esto afecta al desarrollo de la aplicación.

\subsection{Metodología de validación de videojuegos educativos}

Petri y Gresse von Wangenheim, en su trabajo \cite{HowGamesComputingEducationEvaluated}, llevaron a cabo una revisión de la literatura antes de desarrollar el modelo MEEGA+, cuya publicación tuvo lugar en 2018 \cite{meegaplus}. En dicha revisión, analizaron la evaluación de videojuegos educativos relacionados con la computación a partir de una muestra de 3617 artículos. Clasificaron los estudios según la metodología empleada en verdaderos estudios, cuasi-experimentales, no experimentales y ad-hoc.

Los estudios que distribuyen personas de forma aleatoria en distintos grupos se consideraron experimentales. Si un estudio empleaba múltiples grupos o múltiples momentos de medida sin asignación aleatoria, se clasificaba como cuasi-experimental. Los estudios que no utilizaban múltiples grupos, pero que se llevaban a cabo de manera sistemática con estudios de casos, se consideraban no experimentales. Por último, los estudios que no se llevaban a cabo de manera sistemática, ni indicaban cómo se miden resultados, se clasificaban como estudios ad-hoc.

En una reseña de literatura realizada por Calderón y Ruiz \cite{CalderonRuizReviewSeriousGamesEvaluation}, se identificó que la mayoría de los videojuegos educativos se evaluaban en términos de aprendizaje, usabilidad y experiencia de usuario. Además, se señaló que la mayoría de los estudios se realizaban de manera ad-hoc, sin sistematización. Este trabajo afirma que la mayoría de los estudios utilizan cuestionarios y entrevistas como métodos de validación de resultados. Además, se crea una categorización de características que indican la calidad de un juego, la cual fue posteriormente utilizada por \cite{meegaplus} en su cuestionario. Algunos ejemplos de los ítems evaluados incluyen el diseño del juego, la satisfacción del usuario, la usabilidad, la motivación y los resultados del aprendizaje, entre otros.

En su revisión sobre videojuegos serios, Calderón y Ruiz \cite{CalderonRuizReviewSeriousGamesEvaluation} también clasificaron los tipos de procedimientos utilizados para validar estos trabajos, identificando tres tipos: 1) simple; 2) pre/post y 3) pre/post/post. En el primer tipo, los autores llevaron a cabo una sesión con un juego serio, y después de jugarlo, aplicaron los mecanismos de evaluación a los jugadores. En el segundo tipo, la prueba tenía dos etapas de evaluación, una antes del juego serio y otra después, estableciendo así el conocimiento anterior a la prueba. Para el tercer tipo, pre/post/post, además de las pruebas pre y post, se realizaba una prueba semanas después para analizar los niveles de retención. En cuanto a las frecuencias, 50 de 89 de estos estudios utilizaron una metodología simple y el 55\% lo hizo con tamaños de muestra inferiores a 40 \cite{CalderonRuizReviewSeriousGamesEvaluation}.

Basándose en estas conclusiones, Petri y Christiane Gresse von Wangenheim, en su revisión de literatura sobre videojuegos educativos \cite{HowGamesComputingEducationEvaluated}, afirman que la mayoría de los estudios sobre videojuegos educativos se realizan de manera ad-hoc en términos de diseño de investigación, medición, recolección de datos y análisis. Sin embargo, destacan el modelo MEEGA \cite{meegaplusQualityEvaluationPage}, indicando que ha sido utilizado en otras áreas distintas de la computación, proporcionando mayor sistematización y conferiendo mayor validez científica a los trabajos basados en este modelo.


\subsection{Teoría de Respuesta al Ítem (IRT): Intuición Cualitativa}

La Teoría de Respuesta al Ítem (IRT) es un modelo estadístico utilizado en pruebas globales, como exámenes de inglés como lengua extranjera y programas de evaluación internacional de estudiantes. Este modelo destaca por su capacidad para evaluar detalladamente propiedades estadísticas de cada ítem (o pregunta) en términos de dificultad y capacidad de diferenciación \cite{Linden2015HandbookOI, IRTShojima2022}.

Los modelos basados en IRT parten de tres suposiciones fundamentales. Primero, establecen una relación entre el rasgo latente que se busca medir y la probabilidad de responder un ítem en una categoría específica, como "en desacuerdo" o "de acuerdo" \cite{CalderonStatisticalIRT}.

El segundo supuesto es que existe una escala continua y unidimensional de habilidad, denotada como $\theta$. Aunque esta suposición es fuerte, hay técnicas para verificar la unidimensionalidad de los datos de prueba \cite{IRTShojima2022}. Para medir múltiples características simultáneamente, existen modelos IRT multidimensionales \cite{Reckase2009MultidimensionalIRT}.

La tercera suposición es la independencia local, que establece que las personas responden de forma independiente a cada ítem o pregunta, sin que la respuesta a una influya en la respuesta a otra \cite{CalderonStatisticalIRT}.

Por definición, $\theta$ está en el rango $]-\infty, \infty [$, pero prácticamente la totalidad de los datos está en el rango aproximado de $]-3, 3[$. Un valor $\theta = 0$ se asume como un nivel promedio \cite{IRTShojima2022}. 

Existen distintos modelos logísticos para evaluar $\theta$. El que se usa en este trabajo es el modelo logístico de 2 parámetros (2PLM o Two-Parameter Logistic Model) y otra variante de tres parámetros. Los parámetros en estos casos buscan representar rasgos asociados a cada ítem o pregunta. El de 2 parámetros incluye dificultad y discriminación, aunque en español se podría interpretar mejor como distinción o graduación, asociado a indicar la probabilidad entre elegir un valor u otro \cite{CalderonStatisticalIRT}.  El valor de $\theta$ se puede aproximar utilizando el método bayesiano EAP (Expected A Posteriori), el cual se calcula utilizando los paquetes de R \textit{mirt} \cite{RMIRT} y \textit{mirtCAT} \cite{RPackageMIRTCAT} aplicados en este trabajo. 

El modelo 2PLM se compone del parámetro dificultad $b_j$ y de discriminación $a_j$ asignados para cada pregunta. Este último representa la pendiente de la curva e indica en qué medida el ítem diferencia a los examinados con un nivel en el rasgo latente por encima o debajo del parámetro de dificultad \cite{TeoriaRespuestaAlItemPsicologia}. En un modelo con respuestas correctas e incorrectas, un ítem con un parámetro de dificultad mayor $b_j$ será respondido incorrectamente con mayor probabilidad para casi cualquier nivel de $\theta$ \cite{IRTShojima2022}.


\begin{figure}[h]
	\centering
	\includegraphics[scale=.5]{imagenes/IRTparambdifficulty.png}
	\caption{Probabilidad de responder correctamente para distintos valores del parámetro dificultad \textit{b}.}
	\label{ParamDifficulty}
\end{figure}


Es importante destacar que cada ítem discrimina mejor en torno a valores de $\theta$ que sean cercanos al parámetro de locación $b$. Además, un parámetro $a$ indica que el item es un mejor indicador para $\theta$, pero teniendo en consideración que este poder discriminativo funciona mejor cuando $\theta \approx b$. Por esta razón, cada pregunta por separado entrega información acerca del valor $\theta$ que se quiere obtener, de manera que las preguntas deben estar formuladas de tal manera que permitan distinguir distintos niveles del rasgo a evaluar \cite{TeoriaRespuestaAlItemPsicologia}.

Aplicando este modelo a una escala de Likert de formato ordinal, para cada pregunta se indican 4 parámetros b, para diferenciar entre los niveles 1) Muy en desacuerdo; 2) En desacuerdo; 3) Ni en desacuerdo ni de acuerdo; 4) De acuerdo y 5) Muy de acuerdo \cite{TeoriaRespuestaAlItemPsicologia, meegaplusQualityEvaluationPage}. De esta manera, a medida que $\theta$, la calidad de juego es mejor según \cite{meegaplusQualityEvaluationPage}, las respuestas más tenderán a ser del formato Muy de acuerdo, a excepción de algunas preguntas que no se incluyen en la valoración. El script de R utilizado para el cálculo de $\theta$ se encuentra disponible en \cite{meegaplusQualityEvaluationPage}.

Las fórmulas empleadas, demostraciones relacionadas con IRT o mostrar cómo afectan los distintos parámetros al resultado final está fuera del alcance de este estudio, principalmente porque los paquetes de R se encargan de realizar este trabajo. Para entender mejor cómo funciona este modelo por debajo, puede consultarse \cite{IRTShojima2022, CalderonStatisticalIRT}.

\subsection{Motores de videojuegos: Godot}

Existen diversos motores de videojuegos, como Unreal Engine \cite{UE}, Unity \cite{Unity} y Godot \cite{Godot}. Este último resalta por ser completamente gratuito, de código abierto y poseer una comunidad activa en el desarrollo del motor \cite{GodotGithubRepository}. Los motores de videojuegos ofrecen una ventaja significativa al proporcionar un entorno integrado que abarca diversas funcionalidades para diferentes profesionales. Por ejemplo, facilitan la integración de animaciones, efectos visuales, y trabajo con sonido. Además, incluyen bibliotecas incorporadas comúnmente utilizadas en videojuegos, como colisiones, texturizado, o composición de elementos, permitiendo la unificación de la representación visual, auditiva y programática de un personaje. Los motores de videojuegos también posibilitan la exportación de la aplicación a diversas plataformas sin necesidad de modificar el código \cite{GodotExport}.

Godot emplea principalmente dos lenguajes de programación: GDScript, similar a Python, y C\#. Ambos fueron utilizados en este trabajo. GDScript permite prototipar rápidamente debido a su simplicidad y estrecha integración con el motor. Por otro lado, C\# es más rápido, pero algunas características del motor no están completamente integradas en la versión utilizada durante este trabajo, que es la 3.5.2 \cite{GodotCSharpGDDifferences}.

\chapter{Trabajo Relacionado}


\subsection{Videojuegos educativos o serios}

Un videojuego representa una modalidad de aprendizaje activo, donde el proceso de enseñanza no sigue la estructura tradicional que consiste en tener a un profesor exponiendo frente a un estudiante. En este enfoque, el aprendizaje implica la ejecución de pasos y la participación activa del estudiante en actividades que contribuyen a la construcción del conocimiento. La revisión realizada por Hartikainen et al. \cite{active_learning_review} presenta argumentos a favor del aprendizaje activo, destacando mejores resultados, respaldo político y las demandas contemporáneas del entorno laboral, como habilidades de comunicación y la capacidad de aprendizaje autónomo.

Los videojuegos serios, también conocidos como ``Serious Gaming'', constituyen una forma de aprendizaje activo. Bell y Gibson \cite{evaluation_of_games_for_teaching_cs} sostienen que los juegos educativos son más efectivos que las clases tradicionales, lecturas, videos y tareas. Estos autores afirman que el uso de juegos conlleva a una mejora en la retención, conocimiento factual, habilidades basadas en el conocimiento, y autoeficacia. No obstante, recomiendan que los juegos deben complementarse con otras formas de enseñanza y actividades posteriores al juego que permitan a los estudiantes reflexionar sobre la relación entre los juegos y la materia.

\subsection{Análisis de otros autores sobre el potencial y falencias al trabajar con videojuegos educativos}

Bell y Gibson \cite{evaluation_of_games_for_teaching_cs} identificaron y clasificaron 41 videojuegos de Ciencias de la Computación (CS). Uno de ellos, ``Map Coloring", aborda el tema de grafos, en particular el coloreo de grafos. Aunque se realizó una búsqueda del juego hasta la fecha (2022), no se encontró material al respecto.

Kiili y su equipo \cite{using_videogames_maths} analizaron el uso de videojuegos en enseñanza y evaluación en matemáticas a través de los títulos ``Semideus'' y "Wuzzit Trouble". A través de estos, llegaron a la conclusión de que es posible utilizar videojuegos para enseñar y evaluar al mismo tiempo. Además, caracterizaron estadísticamente las diferencias producidas por el uso de videojuegos entre resultados de un pre test y un post test.

En el trabajo de Zhao y Shute \cite{video_game_foster_computational_thinking}, se enlistan ejemplos de videojuegos diseñados para enseñar programación, como Wu's Castle \cite{wuscastle}, CodeCombat \cite{CodeCombat}, CodeSpell \cite{codespells}, y MiniColon \cite{minicolon}. Estos ejemplos emplean programación con texto. No obstante, también existen numerosos referentes que utilizan programación por bloques, como LightBot \cite{LightBot}, Scratch \cite{ scratch, maloney2010scratch}, y RoboBuilder \cite{RoboBuilder}, los cuales simplifican el proceso de aprender una sintaxis relacionada con los lenguajes de programación.

A pesar de esto, el impacto de estos videojuegos no ha sido evaluado en muchos casos. En las instancias documentadas, las muestras suelen ser reducidas, y las evaluaciones se centran principalmente en aspectos cualitativos \cite{video_game_foster_computational_thinking, effectiveness_gbl}.

Petri y otros \cite{meegaplus} coinciden en la falta de sistematización en la evaluación de videojuegos educativos. De hecho, existen juegos serios que ni siquiera se autodenominan o se consideran como tales \cite{evaluation_of_games_for_teaching_cs}. Además, no hay un método estándar para evaluarlos, razón por la cual Petri et al. \cite{meegaplus} desarrollaron el modelo MEEGA+ (Modelo para la Evaluación de Juegos Educativos y Escala EGameFlow.

Entre las deficiencias señaladas al momento de crear videojuegos, se encuentran la carencia de: 1) Definición de un objetivo de evaluación; 2) Diseño de investigación; 3) Plan de medición; 4) Instrumentos de recopilación de datos; y 5) Métodos de análisis de datos. Un ejemplo de falta de sistematización es la práctica común al analizar estas herramientas, que implica comentarios informales por parte del estudiantado \cite{meegaplus}.


\chapter{Marco teórico}

\section{Metodología de validación de videojuegos educativos}

Petri y Gresse von Wangenheim, en su trabajo \cite{HowGamesComputingEducationEvaluated}, llevaron a cabo una revisión de la literatura antes de desarrollar el modelo MEEGA+, cuya publicación tuvo lugar en 2018 \cite{meegaplus}. En dicha revisión, analizaron la evaluación de videojuegos educativos relacionados con la computación a partir de una muestra de 3617 artículos. Clasificaron los estudios según la metodología empleada en verdaderos estudios, cuasi-experimentales, no experimentales y ad-hoc.

Los estudios que distribuían personas de forma aleatoria en distintos grupos se consideraron experimentales. Si un estudio emplea múltiples grupos o momentos de medida sin asignación aleatoria, se clasificaba como cuasi-experimental. Los estudios que no utilizaban múltiples grupos, pero que se llevaban a cabo de manera sistemática con estudios de casos, se consideraban no experimentales. Por último, los estudios que no se llevaban a cabo de manera sistemática, ni indican cómo se miden resultados, se clasificaban como estudios ad-hoc.

En una reseña de literatura realizada por Calderón y Ruiz \cite{CalderonRuizReviewSeriousGamesEvaluation} destacaron que la mayoría de los estudios sobre videojuegos educativos se centran en la evaluación del aprendizaje, la usabilidad y la experiencia del usuario. Además, observaron que la mayoría de estos estudios se llevan a cabo de manera ad-hoc, sin una sistematización adecuada. Se evidenció que los métodos de validación más comunes son cuestionarios y entrevistas. Asimismo, en su trabajo se propuso una categorización de características que indican la calidad de un juego, la cual fue posteriormente utilizada por los autores de MEEGA+ \cite{meegaplus} en su cuestionario. Entre los aspectos evaluados se incluyen: el diseño del juego, la satisfacción del usuario, la usabilidad, la motivación y los resultados del aprendizaje, entre otros.

En su revisión sobre videojuegos serios, Calderón y Ruiz \cite{CalderonRuizReviewSeriousGamesEvaluation} también categorizaron los métodos utilizados para validar estos trabajos en tres tipos: simple, pre/post y pre/post/post. En el primer enfoque, se llevaba a cabo una sesión de juego y luego se aplicaban los mecanismos de evaluación. En el segundo, se realizaban evaluaciones antes y después del juego para medir el conocimiento previo y adquirido. En el tercero, además de las evaluaciones pre y post, se llevaba a cabo una prueba semanas después para evaluar la retención. En cuanto a las frecuencias, 50 de 89 de estos estudios utilizaron una metodología simple y el 55\% lo hizo con tamaños de muestra inferiores a 40 \cite{CalderonRuizReviewSeriousGamesEvaluation}.

Basándose en estas conclusiones, Petri y Christiane Gresse von Wangenheim, en su revisión de literatura sobre videojuegos educativos \cite{HowGamesComputingEducationEvaluated}, afirman que la mayoría de los estudios sobre videojuegos educativos se realizan de manera ad-hoc en términos de diseño de investigación, medición, recolección de datos y análisis. Sin embargo, destacan el modelo MEEGA+ \cite{meegaplusQualityEvaluationPage}, indicando que ha sido utilizado en otras áreas distintas de la computación, proporcionando mayor sistematización y confiriendo mayor validez científica a los trabajos basados en este modelo.

\section{Teoría de Respuesta al Ítem (IRT): Intuición Cualitativa}

La Teoría de Respuesta al Ítem (IRT) es un modelo estadístico utilizado en pruebas globales, como exámenes de inglés como lengua extranjera y programas de evaluación internacional de estudiantes. Este modelo destaca por su capacidad para evaluar detalladamente propiedades estadísticas de cada ítem (o pregunta) en términos de dificultad y capacidad de diferenciación \cite{Linden2015HandbookOI, IRTShojima2022}.

Los modelos basados en IRT parten de tres suposiciones fundamentales. Primero, establecen una relación entre el rasgo latente que se busca medir y la probabilidad de responder un ítem en una categoría específica, como ``en desacuerdo'' o ``de acuerdo'' \cite{CalderonStatisticalIRT}.

El segundo supuesto es que existe una escala continua y unidimensional de habilidad, denotada como $\theta$. Aunque esta suposición es fuerte, hay técnicas para verificar la unidimensionalidad de los datos de prueba \cite{IRTShojima2022}. Para medir múltiples características simultáneamente, existen modelos IRT multidimensionales \cite{Reckase2009MultidimensionalIRT}.

La tercera suposición es la independencia local, que establece que las personas responden de forma independiente a cada ítem o pregunta, sin que la respuesta a una influya en la respuesta a otra \cite{CalderonStatisticalIRT}.

Por definición, $\theta$ está en el rango $]-\infty, \infty [$, pero prácticamente la totalidad de los datos está en el rango aproximado de $]-3, 3[$. Un valor $\theta = 0$ se asume como un nivel promedio \cite{IRTShojima2022}. 

El modelo de 2 parámetros (2PLM o Two-Parameter Logistic Model) considera la dificultad y la discriminación (difficulty \& discrimination) de cada ítem. La dificultad se refiere a qué tan difícil es un ítem en relación con otros; mientras que la discriminación se refiere a la capacidad del ítem para distinguir entre participantes con diferentes niveles de habilidad. 

Existen distintos modelos logísticos para evaluar $\theta$. El que se usa en este trabajo es el modelo logístico de 2 parámetros y otra variante de tres parámetros. Los parámetros en estos casos buscan representar rasgos asociados a cada ítem o pregunta. El de 2 parámetros incluye dificultad y discriminación, aunque en español se podría interpretar mejor como distinción o graduación, asociado a indicar la probabilidad entre elegir un valor u otro \cite{CalderonStatisticalIRT}.  El valor de $\theta$ se puede aproximar utilizando el método bayesiano EAP (Expected A Posteriori), el cual se calcula utilizando los paquetes de R \textit{mirt} \cite{RMIRT} y \textit{mirtCAT} \cite{RPackageMIRTCAT} aplicados en este trabajo. 

El modelo 2PLM se compone del parámetro dificultad $b_j$ y de discriminación $a_j$ asignados para cada pregunta. Este último representa la pendiente de la curva e indica en qué medida el ítem diferencia a los examinados con un nivel en el rasgo latente por encima o debajo del parámetro de dificultad \cite{TeoriaRespuestaAlItemPsicologia}. En un modelo con respuestas correctas e incorrectas, un ítem con un parámetro de dificultad mayor $b_j$ será respondido incorrectamente con mayor probabilidad para casi cualquier nivel de $\theta$ \cite{IRTShojima2022}.


\begin{figure}[h]
	\centering
	\includegraphics[scale=.5]{imagenes/IRTparambdifficulty.png}
	\caption{Probabilidad de responder correctamente para distintos valores del parámetro dificultad \textit{b}. Fuente: \cite{IRTShojima2022}.}
	\label{ParamDifficulty}
\end{figure}


Es importante destacar que cada ítem discrimina mejor en torno a valores de $\theta$ que sean cercanos al parámetro de locación $b$. Además, un parámetro $a$ indica que el item es un mejor indicador para $\theta$, pero teniendo en consideración que este poder discriminativo funciona mejor cuando $\theta \approx b$. Por esta razón, cada pregunta por separado entrega información acerca del valor $\theta$ que se quiere obtener, de manera que las preguntas deben estar formuladas de tal manera que permitan distinguir distintos niveles del rasgo a evaluar \cite{TeoriaRespuestaAlItemPsicologia}.

Aplicando este modelo a una escala de Likert de formato ordinal, para cada pregunta se indican 4 parámetros b, para diferenciar entre los niveles 1) Muy en desacuerdo; 2) En desacuerdo; 3) Ni en desacuerdo ni de acuerdo; 4) De acuerdo y 5) Muy de acuerdo \cite{TeoriaRespuestaAlItemPsicologia, meegaplusQualityEvaluationPage}. De esta manera, a medida que $\theta$ es mayor, la calidad de juego es mejor según \cite{meegaplusQualityEvaluationPage}. Un juego de buena calidad tenderá a recibir respuestas positivas, como  ``Muy de acuerdo''. El script de R utilizado para el cálculo de $\theta$ se encuentra disponible en \cite{meegaplusQualityEvaluationPage}.

Las fórmulas empleadas, demostraciones relacionadas con IRT o mostrar cómo afectan los distintos parámetros al resultado final está fuera del alcance de este estudio, principalmente porque los paquetes de R se encargan de realizar este trabajo. Para entender mejor cómo funciona este modelo por debajo, puede consultarse \cite{IRTShojima2022, CalderonStatisticalIRT}.

\section{Motores de videojuegos: Godot}

Existen diversos motores de videojuegos, como Unreal Engine \cite{UE}, Unity \cite{Unity} y Godot \cite{Godot}. Estos programas ofrecen una ventaja significativa al proporcionar un entorno integrado que abarca diversas funcionalidades para diferentes profesionales. Por ejemplo, facilitan la integración de animaciones, efectos visuales y de sonido. Además, incluyen bibliotecas incorporadas comúnmente utilizadas en videojuegos tales como: colisiones, texturizado o composición de elementos; permitiendo la unificación de la representación visual, auditiva y programática de un personaje. Los motores de videojuegos también posibilitan la exportación de la aplicación a diversas plataformas sin necesidad de modificar el código \cite{GodotExport}. Con todo, Godot destaca por ser completamente gratuito, de código abierto y poseer una comunidad activa en el desarrollo del motor \cite{GodotGithubRepository}.

Godot emplea principalmente dos lenguajes de programación: GDScript, similar a Python, y C\#. Ambos fueron utilizados en este trabajo. GDScript permite prototipar rápidamente debido a su simplicidad y estrecha integración con el motor. Por otro lado, C\# es más rápido, pero algunas características del motor no están completamente integradas en la versión de Godot utilizada durante este trabajo. Se utilizaron dos versiones, se partió en la 3.5.2 y se terminó usando la 3.6 \cite{GodotCSharpGDDifferences}.

\chapter{Diseño de la Solución}

% Contenidos basados en https://repositorio.uchile.cl/bitstream/handle/2250/191381/TablaConten.pdf?sequence=2&isAllowed=y
La aplicación creada, llamada IGACSE (Interactive Graph Algorithms for Computer Science Education) de ahora en adelante, es 
propuesta y diseñada en base a diversos requerimientos y necesidades de los estudiantes de computación.

\section{Referentes}

Se tiene como supuesto de que muchas veces los estudiantes creen entender cómo funciona cierto código o algoritmo, pero no lo 
aplican paso a paso con papel y lápiz o en su imaginación propia. Cuando se asigna como ejercicio realizar el procedimiento 
siguiendo cada instrucción rigurosamente, los estudiantes no lo realizan, o lo hacen de forma incorrecta sin darse cuenta.

El videojuego logrado debe evitar la mecanización por parte del estudiante, obligándolo a revisar exhaustivamente cada paso
relacionado con el algoritmo, de manera que no sé por sentado que se entiende el código o de que se comprendió a cabalidad.

Por otra parte, para que enseñar conceptos relacionados con programación, lo ideal es buscar usuarios que sepan cómo funciona, pero
que no tengan tanta experiencia y que no necesariamente visualicen cómo funciona instrucción por instrucción. Por lo mismo, se 
propone basarse en un modelo similar a los debuggers, donde se puede ver el estado de las variables en cada paso, así como la 
instrucción por ejecutarse y el resultado de la misma.

\section{Descripción de la aplicación al momento de la realización de los experimentos}

El videojuego se separa en distintos niveles. Consta de un menú principal, tutoriales y niveles jugables. 
En el menú principal se puede seleccionar el nivel a jugar o un modo historia. En el modo historia, se juegan todos los 
niveles en orden, y se desbloquean a medida que se avanza.

En los tutoriales se enseñan los conceptos básicos de grafos, pero sin indicarle al usuario explícitamente qué es un grafo. 
Además, se enseñan conceptos de jugabilidad, cómo explorar o seleccionar un nodo, cómo navegar en el código ejecutando instrucciones, 
y cómo contestar preguntas del tipo Sí/No cuando el código tiene una instrucción que incluya un if.

Finalmente, se presentan los niveles jugables, que corresponden a los dos algoritmos que se buscan enseñar: BFS (Breadth First Search) y 
DFS (Depth First Search). En estos niveles no se presenta historia y hay menos ayuda para el usuario.

% Mostrar aquí muchas imágenes del resultado por partes.
% Mostrar un diagrama de flujo dentro del juego. Menu => (Tutorial o Niveles). Tutorial => Niveles. Niveles => Créditos.
% Mostrar cómo funcionan los tutoriales, cada uno. Dar muestras de la historia o narrativa presentada.
\subsection{Diagrama de flujo de juego}

\subsection{Menú principal}

\subsection{Tutoriales}

\subsection{Niveles jugables}

\subsection{Narrativa}


% utiliza la analogía de planetas y rutas espaciales. Esta decisión se basó en que los videojuegos suelen poseer una narrativa, 
% la cual lo hace más atractivo. Además, una narrativa ayuda a aterrizar un concepto abstracto como pueden ser los grafos.

\section{Requisitos de usuario}
\section{Requisitos de software}
\section{Requisitos no funcionales}
\section{Proceso de diseño}
\subsection{Diseño de la interfaz de usuario}
\subsection{Diseño de mecánicas de juego}


\section{Arquitectura de software}

\section{Implementación de la solución}
% Mencionar que se usó Godot, GDScript, bla bla bla



\chapter{Diseño experimental: Metodología}

El propósito de este trabajo fue poner a prueba la hipótesis de que un videojuego educativo puede enseñar algoritmos relacionados con grafos y que puede ser percibido positivamente por parte de los estudiantes.

Para ello, se llevó a cabo un experimento en el que se solicitó a voluntarios que respondieran ciertas preguntas después de probar la aplicación. Este enfoque se considera una metodología simple según \cite{HowGamesComputingEducationEvaluated}, es importante destacar que no es el más adecuado para este tipo de estudios, ya que la mayor validez científica se obtiene con una metodología pre/post/post y un tamaño muestral de 40 individuos o más.

Sin embargo, implementar este tipo de estudios requiere más tiempo por parte de los voluntarios, así como mayores incentivos para su participación y respaldo institucional. Para ejemplificar esto último, la participación en este estudio podría formar parte de una actividad pedagógica en un curso y ser obligatoria.

Se sugieren buenas prácticas metodológicas en \cite{Rogers2002InteractionDesign, MeegaPlusManual, HowGamesComputingEducationEvaluated}, donde se establecen ideas primordiales para identificar las dificultades a resolver. Además, se destaca la importancia de realizar validaciones con tamaños muestrales significativos y representativos del usuario al que está destinado el trabajo. Asimismo, se sugiere aprovechar tecnologías como eye-tracking y telemetría. Sin embargo, la obtención de voluntarios para participar en las pruebas fue un desafío, sobre todo considerando que aquellos que participaran en una prueba inicial no podrían formar parte de las pruebas finales, que se consideraban más relevantes.

Se sugieren buenas prácticas metodológicas en \cite{Rogers2002InteractionDesign, MeegaPlusManual, HowGamesComputingEducationEvaluated}, donde se establecen ideas primordiales para identificar las dificultades a resolver. Además, se destaca la importancia de realizar validaciones con tamaños muestrales significativos y representativos del usuario al que está destinado el trabajo. Asimismo, se sugiere aprovechar tecnologías como eye-tracking y telemetría. Sin embargo, la obtención de voluntarios para participar en las pruebas fue un desafío, sobre todo considerando que aquellos que participaran en una prueba inicial no podrían formar parte de las pruebas finales, que se consideraban más relevantes.


\section{Fases del trabajo de investigación}

El trabajo se dividió en tres fases, las cuales constan de distintas preguntas después de probar la aplicación.


\subsection{Fase exploratoria}

En esta primera fase, el objetivo fue recopilar retroalimentación y opiniones de usuarios de manera abierta, utilizando metodologías de pensamiento en voz alta con personas experimentadas en grafos. Se optó por esta metodología por dos razones. En primer lugar, si un usuario conoce el algoritmo y la estructura de datos pero no comprende el videojuego, entonces existen problemas de usabilidad, justificando el enfoque inicial con personas expertas.

En segundo lugar, los usuarios expertos tienden a expresarse de manera más natural cuando pueden emitir opiniones en el momento, por lo que se prefirió evitar encuestas o escalas posteriores a la experiencia. Resulta crucial que los expertos expresen sus impresiones a medida que los elementos del videojuego aparecen en pantalla y no después de la experiencia, para comprender mejor lo experimentado al usar la aplicación por primera vez. Hay animaciones que deben captar la atención en el momento, como la pista visual del ratón indicando al usuario que haga clic izquierdo en un planeta.

Este desarrollo fue iterativo. Primero se hacían cambios a la aplicación basados en los últimos comentarios recibidos. Luego, se le mostraba la aplicación a usuarios expertos, estos daban su opinión y se volvían a aplicar los cambios. Este proceso se hizo 8 veces con el curso ``CC7970 - Trabajo de Tesis I''.


\subsection{Fase de evaluación académica y percepción de usuario}

En esta fase, se logró la participación completa de una muestra de 15 personas, a quienes se les ofreció un incentivo monetario para evitar sesgos asociados a la voluntariedad y para aumentar la participación \cite{Marinescu2018IncentivesCR, Dallmeyer2023ToPayOrNot}.

Para iniciar esta etapa, se realizó una convocatoria voluntaria en el foro de las tres secciones de Algoritmos y Estructuras de Datos de la Universidad de Chile durante las primeras dos semanas del semestre de primavera del año 2023. Se ofreció una remuneración a todas las personas que completaran la experiencia, la cual tenía una duración promedio de 45 minutos. 

Las personas que rinden este curso suelen estar en su quinto semestre de la carrera, aunque algunas también lo rinden de forma más tardía en sus carreras, sobre todo cuando son de otras especialidades como Ingeniería Eléctrica o Industrial. Estos suelen tener nociones de programación, como declaración de variables y control de flujo, pero no necesariamente sobre estructuras de datos. Estos estudiantes rondan los 21 a 23  años de edad.

La convocatoria se realizó por dos medios. Por una parte, se escribió un mensaje en la comunidad de Telegram invitando a la gente del curso a participar del estudio, y por otra parte, el equipo docente de cada sección del curso CC3001 - Algoritmos y Estructuras de Datos, escribió un mensaje en el foro invitando a sus estudiantes a participar.


La experiencia constaba de tres partes: realizar la prueba de usuario, completar un formulario que utilizaba la escala de Likert basado en el modelo MEEGA+ \cite{meegaplus}, y responder a una prueba escrita basada en exámenes previos del curso CC3001 antes mencionado.

El modelo sistemático MEEGA+ \cite{meegaplus} está diseñado para evaluar videojuegos educativos, buscando evaluar la percepción de la calidad de un videojuego desde la perspectiva del estudiante en el contexto de la enseñanza de la computación. El formulario utilizado en este estudio, basado en MEEGA+, se encuentra en el anexo A.

La medición del rendimiento académico se lleva a cabo mediante una prueba escrita con dos preguntas, basadas en una pregunta de un examen del ramo de Algoritmos y Estructuras de Datos de la misma facultad. La prueba escrita se encuentra en el anexo anexo B.


\subsection{Encuesta libre}

% Anotar aquí antes de que termine la tesis cuánta gente fue en total.

Con el objetivo de ampliar el estudio y aumentar el tamaño de la muestra, se llevó a cabo una tercera experiencia abierta al público en general con conocimientos en programación. Se emitió una invitación en diversas comunidades de videojuegos para probar el juego educativo y completar el formulario. Dado que las experiencias podían variar significativamente, se incorporó una pregunta sobre el nivel de experiencia en programación para permitir la segmentación. El formulario utilizado también se basó en el modelo MEEGA+, pero las respuestas recolectadas se almacenaron en una base de datos separada del grupo anterior. Aquí se mostraron dos versiones, una en español y otra en inglés para aumentar el tamaño de la muestra. En total, 12 personas participaron de esta experiencia.

\chapter{Resultados}

\subsection{Prueba académica}

Las 15 personas que hicieron la prueba tuvieron ambas respuestas correctas, identificando correctamente el recorrido BFS y DFS en cada caso.


\subsection{Formulario MEEGA+: Percepción de usuario}

% Hablar de la suma de puntaje y puntaje promedio para los dos casos: De fase exploratoria y fase posterior.
Las respuestas al formulario se encuentran en: \href{https://docs.google.com/spreadsheets/d/1DlxUT5o-nRknYXGtFWqkXX2ZkuEZJZsow4DbW6bUTh8/edit?usp=sharing}{https://docs.google.com/spreadsheets/d/1DlxUT5o-nRknYXGtFWqkXX2ZkuEZJZsow4DbW6bUTh8/edit?usp=sharing}


Debería poner aquí los gráficos o en un anexo?
% Poner acá gráficos y demás


\chapter{Discusión}

% \cite{Yu2020TheEffectsOfEducationGames} Difficulty in Determining the Effectiveness of Educational Games
% Mencionar aquí que desde ya se ven varias debilidades en general en el área STEM, además que no es aceptado.
% Suggestions for Educational Game Designers
% While many studies have been committed to the effect of educational games on
% learning, few of them have shed light on their design (Fanfarelli, 2020). Game
% features, e.g. perceived usefulness, ease of use, and goal clarity, could increase
% student engagement and improve the enjoyment of games, which should be
% stressed by game designers (Y. C. Wang et al., 2017). 
% Indicar que este videojuego no fue hecho por un diseñador o fue pensado fuertemente en su diseño
% Y que se hizo énfasis en la metolodogía de: línea de código o instrucción con un correlato de acción por parte del jugador
% en el videojuego.

\section{Fortalezas de la metodología aplicada}

\section{Debilidades de la metodología aplicada}

\section{Mejoras en una experiencia futura}

\section{Trabajo futuro}




\chapter{Conclusión}

El videojuego logra su objetivo de enseñar los algoritmos de BFS y DFS, como evidencian los resultados obtenidos por estudiantes de informática de la Universidad de Chile. Estos estudiantes demostraron comprender los algoritmos, así como la naturaleza de los grafos, al responder de manera acertada a una prueba asociada al juego.

%%% TODO: VERIFICAR %%%
Además, el juego ha recibido una evaluación positiva y se considera, según el modelo MEEGA+ \cite{meegaplus}, como un juego de calidad. A pesar de las mejoras identificadas durante el proceso, el juego ha sido bien recibido por los estudiantes, y su percepción positiva puede influir beneficiosamente en la motivación y, por ende, en los resultados académicos. Aunque se reconocen las áreas de mejora, este trabajo sienta las bases para futuras investigaciones, especialmente en relación con el diseño de videojuegos educativos y su impacto en la educación en informática. La metodología utilizada aquí podría ser profundizada y perfeccionada en trabajos futuros para evaluar de manera más precisa la efectividad y comprensión de los conceptos enseñados mediante este enfoque lúdico.


% Énfasis en Resultados Positivos: Resalta de manera más específica los resultados positivos obtenidos, como el hecho de que los estudiantes lograron comprender los algoritmos y la estructura de los grafos, así como aprobar la prueba asociada. Esto ayuda a consolidar la efectividad del videojuego como herramienta educativa.

% Impacto en la Motivación: Puedes profundizar en cómo la percepción positiva del juego podría influir en la motivación de los estudiantes. ¿Se notó un aumento en el interés por el tema? ¿Hubo comentarios que sugieran un cambio en la actitud hacia el aprendizaje?

% Relación con Otros Métodos de Enseñanza: Considera mencionar cómo este enfoque se integra o complementa con otros métodos de enseñanza tradicionales. ¿El juego podría servir como apoyo en determinadas etapas del aprendizaje?

% Desafíos y Lecciones Aprendidas: Brevemente, comparte los desafíos encontrados durante el desarrollo e implementación del videojuego, así como las lecciones aprendidas. Esto podría incluir aspectos técnicos, metodológicos o de diseño que podrían ser útiles para otros investigadores o desarrolladores en proyectos similares.

% Relevancia en la Educación Actual: Haz hincapié en la relevancia del juego en el contexto educativo actual. ¿Cómo se alinea con las tendencias y necesidades de la educación en informática?

% Posibles Extensiones del Trabajo: Además de lo que ya has mencionado sobre futuras investigaciones, podrías sugerir posibles extensiones o mejoras específicas para el juego en función de los comentarios y resultados obtenidos.

% ver https://www.overleaf.com/learn/latex/Glossaries
% \input{glosario.tex} % opcional

\nocite{*}
\bibliographystyle{plain}
\bibliography{bibliografia}

% opcional ...
% \begin{appendices}w
\chapter{Anexo A: Formulario basado en MEEGA+} \label{AnexoA}

\newgeometry{margin=0.5in}

\begin{table}[h]
\centering
\caption{Formulario basado en MEEGA+. Se eliminó la dimensión social y no se incluyeron los ítems que preguntan por los objetivos particulares del juego.}
\label{TablaFormularioMEEGA}

\begin{tabular}{|c|l|} % borde | columna fuente centrada | borde | columna fuente a la izq | 
\hline % Linea horizontal inicial
\textbf{Notación} & \textbf{Pregunta} \\\hline % Separa el inicio
Q1       & El diseño de juego es atractivo \\ 
Q2       & La fuente de texto y los colores están bien combinados y son consistentes \\
Q3       & Tuve que aprender cosas antes de poder jugar \\
Q4       & Aprender a jugar se me hizo fácil \\
Q5       & Las reglas del juego son claras y fáciles de entender \\
Q6       & Las fuentes (su tamaño y estilo) usadas son fáciles de leer \\
Q7       & Los colores usados en el juego son significativos \\
Q8       & Cuando vi el juego por primera vez, tuve la sensación de que sería fácil para mí \\
Q9       & La estructura me ayudó a tener confianza en que aprendería con este juego \\
Q10      & Este juego es desafiante en la medida justa \\
Q11      & El juego ofrece nuevos desafíos a un ritmo adecuado \\
Q12      & El juego no se vuelve monótomo a medida que se avanza \\
Q13      & Completar las tareas del juego me provocó satisfacción \\
Q14      & Pude avanzar en el juego gracias a mi esfuerzo personal \\
Q15      & Siento satisfacción con lo aprendido en este juego \\
Q16      & Recomendaría este juego a mis colegas \\
Q17      & Me entretuve con el juego \\
Q18      & Algún elemento del juego me hizo sonreír \\
Q19      & Había algo interesante al inicio del juego que llamó mi atención \\
Q20      & Estaba tan envuelt@ en la tarea propuesta por el juego que perdí la noción del tiempo \\
Q21      & Me olvidé de mi entorno físico mientras jugaba \\
Q22      & Los contenidos del juego son relevantes para mis intereses \\
Q23      & Es claro ver que los contenidos del juego se relacionan a cierta materia de la carrera \\
Q24      & Este juego es adecuado para enseñar el contenido \\
Q25      & Prefiero aprender con este juego en vez de aprender con otros métodos de enseñanza \\
Q26      & El juego contribuyó a mi aprendizaje \\
Q27      & El juego me permitió aprender de forma eficiente en comparación con otras actividades \\
& \\
\hline
\end{tabular}
\end{table}

\restoregeometry


\chapter{Anexo B: Prueba de conocimiento de grafos} \label{AnexoB}
\textbf{Test para verificar aprendizaje de BFS y DFS}

Considere el siguiente grafo:

\begin{figure}[h]
        \centering
    \includegraphics[width=0.6\textwidth]{imagenes/GrafoPrueba.png}
\end{figure}

Muestre el recorrido de un grafo desde el nodo a utilizando distintos los dos algoritmos de búsqueda vistos. \textbf{Agregue una numeración al lado de sus arcos para indicar el orden} en que se explorando.
Por ejemplo, en el siguiente grafo, si se recorre \textbf{a}, \textbf{b} y \textbf{c} partiendo desde \textbf{a, pasando por b y llegando a c}, el resultado esperado es el siguiente:

\begin{figure}[h]
        \centering
        \includegraphics[ width=0.5\textwidth]{imagenes/GrafoPruebaEjemplo.png}
\end{figure}

\newpage 
1.	Aplique el algoritmo \textbf{Búsqueda en Profundidad (DFS / Depth First Search)} para mostrar cómo se recorre un grafo partiendo desde el nodo a.

\begin{figure}[htbp]
        \centering
    \includegraphics[width=0.7\textwidth]{imagenes/GrafoPrueba.png}
\end{figure}

2.	Repita lo anterior usando el algoritmo de \textbf{Búsqueda en Achura (BFS / Breadth First Search)}

\begin{figure}[htbp]
        \centering
    \includegraphics[width=0.7\textwidth]{imagenes/GrafoPrueba.png}
\end{figure}

%\newgeometry{margin=0.1in}
%\begin{figure}[htbp]
    %    \centering
    %    \fbox{\includegraphics[page=1, width=0.7\textwidth]{imagenes/TestParaVerificarAprendizaje.pdf}}
    %    \caption{Your caption here}
    %    \label{fig:yourlabel}
    %\end{figure}
%\restoregeometry

\chapter*{Anexo C: Respuestas de pruebas de usuario}\label{AnexoC}

\subsection*{Prueba de usabilidad en conjunto con prueba académica}
Q1;Q2;Q3;Q4;Q5;Q6;Q7;Q8;Q9;Q10;Q11;Q12;Q13;Q14;Q15;Q16;Q17;Q18;Q19;Q20;Q21;Q22;Q23;Q24;Q25;Q26;Q27;Q28;Q29;Q30;Q31



\subsection*{Prueba de usabilidad en formato libre para cualquier usuario}
\chapter{Anexo D: Aprobación de Comité de Ética}\label{AnexoD}
% Poner aquí el PDF del comité de ética


\chapter*{Anexo E: Script en R que permite obtener el valor indicador de la calidad de juego}\label{AnexoE}

\newgeometry{margin=0.5in}


\begin{lstlisting}[language=R, caption=Script en R para evaluar el videojuego]
library(data.table)
library(mirt)
library(mirtCAT)
# CALCULATING THE SCORES
responses <- fread(input = "EXTRA.csv", header = T)
pars <- fread(input = "param.csv", header = T)
pars <- data.frame(
   a1=pars$a1, d1=pars$d1, d2=pars$d2, d3=pars$d3, d4=pars$d4)
modelo <- generate.mirt_object(
   parameters = pars, itemtype = "graded", min_category=1)
escore <- fscores(
   object = modelo, method = "EAP", response.pattern = used_responses)
SCORE_TRI <- escore[ , 1]
# Theta is adjusted to improve interpretability.
final_theta <- SCORE_TRI * 0.15 + 0.5
mean(final_theta) # Theta is = 0.6055371
\end{lstlisting}

\restoregeometry

\chapter*{Anexo F: Comentarios abiertos entregados por los voluntarios}\label{AnexoF}

\newgeometry{margin=0.5in}

% 1.1
Las siguientes tablas de comentarios muestran las respuestas del primer grupo, aquel que realizó la prueba académica.

\begin{table}[h]
   \centering
   \caption*{\textit{Comentarios del grupo que realizó la prueba académica frente a la pregunta: Nombra aspectos fuertes del juego}}
   \begin{tabular}{|p{\linewidth}|}
   \hline
   \textbf{Comentario} \\\hline
   Es una forma distinta de aprender. Es interactivo. Me gusta que sea de prueba y error. \\\hline
   Enseña de manera correcta el algoritmo \\\hline
   Era intuitivo y fácil de entender \\\hline
   El aspecto que me llamó mucho la atención y lo destaco como un aspecto fuerte del juego, fue la forma interactiva de mostrar un código de programación y conceptos detrás de estos mismos. \\\hline
   
   De esa forma, se puede visualizar de mejor manera la dinámica detrás de un algoritmo (que muchas veces eso llega a ser un problema a la hora de analizar su funcionamiento, en especial si son algoritmos más complejos) \\\hline
   Me parece que al ser interactivo se entienden mejor los conceptos, además los movimientos de la nave ayudan a entender gráficamente cómo funciona el algoritmo \\\hline
   El diseño y la historia planteada son atractivos, como también la jugabilidad, es idónea para adquirir habilidades de programación. \\\hline
   
   Es intuitivo y no tan difícil de entender la idea, el diseño de niveles y la interfaz de usuario \\\hline
   Incorporar animales tiernos y buena música aporta mucho a la satisfacción y motivación de terminar los niveles \\\hline
   El acompañamiento visual de las líneas del código ayuda mucho al entendimiento del algoritmo sin tener la necesidad de explicar textualmente la función de cada uno \\\hline
 Me gustó poder saltarme las instrucciones porque no me gusta leerlas, menos si es un juego. Me gustaron los colores, llama la atención a pesar de ser un juego sin mucho detalle \\\hline
    La atmósfera del juego permite conectar fácilmente con la experiencia de aprendizaje, además su temática me parece bastante atractiva \\\hline
    Es entretenido y super fácil de seguir las instrucciones \\\hline
    Me gustó que agregue una imagen sobre el contenido de programación que se tratará en cada nivel, eso me facilita el aprendizaje y la idea de lo que estoy realizando. Además, el juego ayuda bastante si te equivocas o no realizaste una acción (como seleccionar el nodo u otros) \\\hline
    Es interesante que sea de programación y astronomía/ es desafiante \\\hline
    Es un juego entretenido, los gráficos llaman la atención, interactivo y uno puede quedarse pegado jugando \\\hline
    Las gráficas son muy bonitas y atractivas, invitan a jugar  \\
   \hline
   \end{tabular}
\end{table}

\restoregeometry

% 1.2

\newgeometry{margin=0.5in}
\begin{table}[h]
   \centering
   \caption*{\textit{Comentarios a la pregunta ``indica una o más sugerencias para mejorar el juego''. Grupo que realizó la prueba académica.}}
   \begin{tabular}{|p{\linewidth}|}
   \hline
   \textbf{Comentario} \\\hline
   Mejorar colocación de títulos. \\ \hline
   Hacerlo menos tedioso, quizás agregar alguna música acorde, poner el código de manera más amigable en vez de un código tan seco, que el grafo inicial sea más pequeño; \\ \hline
   Quizás se podría explicar un poco más de qué significa el código de la derecha para gente que no tiene ni idea de programación. Que es un for, un while y demás, explicar cómo se recorre y por qué; \\ \hline
   Agregar muchos más niveles que formen una gran escalera progresiva de contenidos a aprender. También podría sugerir que en el momento que la persona no interactúe con el juego en un tal momento del algoritmo, el juego diga mini hints que ayuden a pensar a la persona de lo que debería hacer en donde está estancado. Tal vez que sea más largo o que tenga más ejercicios por algoritmo; \\ \hline
   La tipografía de la letra y los colores para facilitar la lectura, hay números que se confunden con letras y requieren contraste con el resto del juego, un buen ejemplo de esto sería el final fantasy donde el texto es solamente un fondo negro con letras blancas los controles, al momento de agregar los planetas yo pensaba que era apretar r no más pero había que poner el mouse sobre el planeta cambiar la letra no me gustaba; \\ \hline
   Las palabras del texto explicativo deberían resaltar las instrucciones centrales. \\ \hline
   Sería positivo incorporar un botón de ayuda, en algunas partes sentí que me quedé atrapado.; \\ \hline
   Las instrucciones del principio deberían ser entregadas de una manera más simple y directa. Quizás un cuadro con una lista de todo lo que hay que hacer al inicio podría ayudar.; \\ \hline
   Mejorar la fuente de texto, se me complicó en ciertos casos leer algunas letras.; \\ \hline
   Podría ser la música que era un poco desesperante y se podría hacer un pequeño tutorial explicando las funciones que se utilizan; \\ \hline
   Si el juego busca enseñar programación, sería bueno colocar un módulo donde se pueda explicar mejor el código o dar alguna explicación de ciertas funciones que no sabía qué hacían pero igual tenía que apretar espacio para pasarlas (podría ser una pequeña explicación al pasar el mouse por sobre el texto). Quizás no era de importancia para el juego, pero como no conozco mucho el lenguaje con el que se utilizó e igual sé programar, sería interesante poder conocer qué significan tales funciones.; \\ \hline
   Encuentro que el juego se puede completar sin entender del todo la diferencia entre queues y stacks, solo mecanizando el seguir las instrucciones del panel derecho. Ambos niveles se diferencian en que el primero es mucho más rápido de recorrer y mecanizar, mientras que el segundo plantea más dificultad, pero no queda completamente claro en cuál se usan queues y en cuál stacks.; \\ \hline
   Quizás faltaron planetas para que se notara más el hecho de acumular elementos en queues o stacks y la forma en que estos se extraen, o quizás sería más sencillo entender estos conceptos si el juego solo consistiera en clickear planetas para ir recorriendo los mapas, sin necesidad de apretar teclas como espacio, R, W o S, ya que a la larga esta mecánica distrae un poco de entender realmente cómo están operando los algoritmos. \\ \hline
   \end{tabular}
\end{table}

\restoregeometry

% 1.3
\newgeometry{margin=0.5in}
\begin{table}[h]
   \centering
   \caption*{\textit{Comentarios a la pregunta ``Algún otro comentario?''. Grupo que realizó la prueba académica.}}
   \begin{tabular}{|p{\linewidth}|}
   \hline
   \textbf{Comentario} \\\hline
   Muy buen juego, es a mi gusta una forma divertida de aprender a manejar grafos. \\ \hline
   Muy entretenido, viva los pandas rojos!! \\ \hline
   Muy interesante y entretenida la propuesta de este juego. Espero que se logré desarrollar más y más :3 \\ \hline
   No soy fan del tema espacial pero si me gusto el juego \\ \hline
   muy bueno \\ \hline
   Las palabras del texto explicativo deberían resaltar las instrucciones centrales. \\ \hline
   Hay unos pasos en los que no hay que hacer ninguna acción, por ejemplo el comando s.pop, y al principio eso me confundió un poco por no tener la certeza de cuál era su función dentro el algoritmo. Creo que eso es parte del desafío del juego y se logra comprender gracias al acompañamiento visual y sonoro \\ \hline
   creo que a pesar de ser estudiante de ingeniería  tengo poca noción de la programación, así que para mi al no leer las instrucciones necesite ayuda para terminar el juego. Ayudaría quizás en el lado derecho no poner un código y si una explicación con palabras \\ \hline
   En líneas generales una excelente temática y experiencia. \\ \hline
   Buen juego, entretenido y didactico para aprender la materia de programación. Me gustaria que hubiesen más juegos así para abordar otros contenidos de programación. \\ \hline
   Muy bueno el juego!!! Felicitaciones \\ \hline
   aguanten los pandas! \\ \hline
   \end{tabular}
\end{table}

\restoregeometry





\section*{Comentarios del grupo que participó en el estudio libre sin restricciones}


\newgeometry{margin=0.5in}

\begin{table}[h]
   \centering
   \caption*{\textit{Comentarios a la pregunta: ``Por favor, indica uno o más aspectos fuertes del juego''. Grupo libre.}}
   \begin{tabular}{|p{\linewidth}|}
   \hline % Linea horizontal inicial
   \textbf{Comentario} \\\hline
   1. El ir paso a paso ejecutando los algoritmos ayuda mucho a entender cómo funcionan y cómo se diferencian entre sí. 2. El feedback visual de algunas cosas como la variable seleccionada o el estado actual del stack/queue que se está usando permite siempre saber lo que está ocurriendo, o incluso retomar la ejecución luego de perder la atención un momento. \\\hline

   La musica es bien llamativa y la tematica de la nave buscando los pandas rojos es atractiva. \\\hline

   La idea de representar lo importante de programar gráficamente es muy interesante, la idea de abstraerlo a una temática de exploración espacial también lo es. \\\hline
   
   Me gusta que te obligue a hacer las instrucciones una a una. \\\hline

   El aspecto visual es bueno, la forma retro del código mostrado al lado derecho me gusta. \\

   \hline
   \end{tabular}
\end{table}

\restoregeometry


\newgeometry{margin=0.5in}

\begin{table}[h]
   \centering
   \caption*{\textit{Comentarios a la pregunta ``indica una o más sugerencias para mejorar el juego''. Grupo libre.}}
   \begin{tabular}{|p{\linewidth}|}
   \hline % Linea horizontal inicial
   \textbf{Comentario} \\\hline
   1. El juego se beneficiaría de más feedback visual sobre las cosas seleccionadas/actuales en el algoritmo, como cuales son los nodos vecinos.
   2. Además, un nivel inicial con nodos ordenados de forma que las líneas no se sobrepongan puede ser útil.
   3. Para notar la diferencia entre BFS y DFS, tener una misma configuración de nodos puede ayudar a entender cómo ocurre que ciertos planetas se visitan antes que otros según el algoritmo que se esté usando.
   4. Finalmente, animaciones in-game para explicar los algoritmos pueden funcionar mucho mejor que gifs a pantalla completa. \\\hline


   Algunos de los gifs de fondo que muestran el recorrido de grafos podrían tener mejor calidad, y creo que algunos dialogos se muestran muy lento en comparacion a otros.
   creo que hace falta niveles más profundos en los algoritmos de recorrido de grafos, para entender una dinámica mayor y quizás algún nivel que te dejen solo y por los requisitos de algún tipo, tengas que utilizar alguno de los algoritmos enseñados\\\hline

   Considero que una mejora en la selección de colores y distribuciones espaciales de los elementos en pantalla haría que gráficamente el juego mejorara mucho, y pienso que eso a su vez ayudaría a que fuera mucho más interesante aprender mediante él. Por otro lado, creo que se le debería dar más libertad al jugador, en especial la posibilidad de que se pudiera equivocar y este error afecte en el funcionamiento del programa (explicando gráficamente de alguna manera por qué se equivocó y qué consecuencias tiene eso), entiendo que es algo complicado de implementar pero creo que si es implementado de una manera inteligente puede potenciar mucho el aprendizaje.\\\hline

   Algunas cosas. Hay un bug que cuando aparece una de las imágenes de pandas rojo, el espacio deja de funcionar para avanzar. \\\hline 
   
   Creo que el no explicar con anterioridad las instrucciones es un poco complicado. Me ví haciendo cosas que no entendía y simplemente siguiendo las instrucciones sin ver el algoritmo a gran escala, además estaba camuflado con todos los aspectos del juego. Creo que sería bueno poner un paralelo un poco más formal a medida que se desarrolla el juego, quizá más animaciones para mostrar que el planeta que seleccionas efectivamente se va a la cola, y quizá además del número debería haber una miniatura del planeta en la lista.  \\\hline
   
   Otra cosa que me pasó un par de veces es que sin querer apreté la barra espaciadora y por coincidencia estaba en la opción correcta, por lo que no entendí qué hice y el juego avanzó. Quizá en cada iteración la consola debería reinicial la posición para tener que mover el selector de manera voluntaria y entender el algoritmo. \\\hline
   
   Otro punto es que cuando haces algo bien, hay un sonio que uno termina relacionando con haber hecho bien el paso en el algoritmo. En algún momento el sonido deja de sonar al inicio de los for, lo que es raro porque no sabes si lo hiciste bien o no. \\\hline

   El aspecto de la cola, ver la forma de avanzar más rápido en esos pasos estaría bueno. \\
   \hline
   \end{tabular}
\end{table}

\restoregeometry


\newgeometry{margin=0.5in}

\begin{table}[h]
   \centering
   \caption*{\textit{Comentarios a la pregunta ``Algún otro comentario?''. Grupo libre.}}
   \begin{tabular}{|p{\linewidth}|}
   \hline % Linea horizontal inicial
   \textbf{Comentario} \\\hline
   Es un juego completamente necesario y una buena idea que puede facilitar mucho el aprendizaje de grafos. Gran trabajo
   PD: Tal vez tener alguna forma de configurar niveles custom podría hacer de esta una herramienta de debugging, que puede estar muy bien :D \\\hline
   Me parecio bien entretenido en general \\\hline
   
   Considero admirable el aportar en el aprendizaje de la computación mediante juegos, y muy valiente considerando que no es una opción "popular" con mucha información al respecto \\\hline
   Me gusta la idea pero creo que hay que pulir los controles y las instrucciones \\\hline
   Me gustó el concepto. Se entiende bien el concepto de búsqueda en grafo y queda ``grabado'' al intentarlo varias veces. Eso sí, como mi objetivo era avanzar en el juego lo hice pese a que los ejercicios con la cola a veces eran más complejos en el sentido de avanzar al estar atento a pulsar el botón "espacio", pero en general me gustó el juego y cómo se aprende en el proceso.
   \\\hline
   \end{tabular}
\end{table}

\restoregeometry

\chapter*{Anexo G: Comentarios de algunos expertos}\label{AnexoG}

Estos comentarios han sido anonimizados y aleatoriazados para proteger la identidad de quienes los emitieron, pues se les garantizó anonimato en caso de que sus comentarios se publicaran.
Muchos comentarios requierne más contexto para ser entendidos, puesto que se enmarcaban en medio de una prueba de usuario, frente a cierta interfaz. Por esta razón, se hizo una selección de comentarios y se muestran aquí.


\newgeometry{margin=0.5in}

\begin{table}[h]
   \centering
   \caption*{\textit{ }}
   \begin{tabular}{|p{\linewidth}|}
   \hline
   \textbf{Comentario} \\\hline
   El juego debería ser consistente, si quieres avanzar con SPACE, apreta SPACE para continuar. Siempre. \\\hline
   (Después de presionar un planeta, hace click en uno y no pasa nada) ¿Qué pasa al hacer click en un planeta? mmm, nada. Falta feedback ahí. \\\hline
   
   En un juego siempre hay que tener clara la condición de ganar.
   Cuál es el objetivo para ganar? La misión es siempre la misma? Anda subdividiendo la misión. \\\hline

   Te recomiendo hacer tutoriales de a poco, como Duolingo. AL principio te muestran las mecánicas con ejercicios muy simples, y luego te dejan andar. \\\hline

   Haz tutoriales, es como la técnica del triciclo. Al principio los ayudas, después los dejas andar solos \\\hline

   No recomiendo destacar las cosas solo con color. Agrega movimiento también. Para la gente con daltonismo puede ser complejo. \\\hline

   Trata de que el mouse cambie cuando pasas por un elemento clickeable o seleccionable. \\\hline

   (Respecto a la popup con un if) La fuente del Yes y No está re mala. Parece muy poco real. \\\hline

   Usa una única fuente constante de información. Si me quedo con una, que sea solo una popup. \\\hline

   Aprovecha el movimiento. Las cosas que se mueven llaman mucho más la atención. Si quieres que hagan click en algo, agrégale movimiento. Si quieres que el usuario lea algo, agrégale movimiento \\\hline

   Prueba testear conceptos de a poquito: Quiero testear este tutorial. Qué cosas ven primero? Después, prueba cambiar los colores de los planetas, resaltarlos, vé cómo eso afecta. Pero es importante probar paso por paso, si haces todos los cambios de una sin probar y falla algo, perderás mucho tiempo y no sabrás exactamente qué falló. \\\hline

   Busca conocer bien a tu público objetivo. Si son gente que está dado Algoritmos y Estructuras de Datos, ve qué edad tienen, qué tipo de animaciones llaman más su atención (...). \\\hline

   El núcleo del juego se debe tratar de que estás siguiendo las instrucciones (...). Lo ideal es no explicitarlo y que se dé a entender por sí solo. \\\hline

   
   Confía en el conocimiento de tu usuario. Gente universitaria que juega
   juegos. Selecciona este nodo y agrégalo a la variable. Que el usuario
   descubra a través de la interfaz cómo se hace eso
   Si la interfaz está bien hecha, el usuario debería encontrar cómo hacerlo. \\\hline

   Hay un concepto que se llama la ceguera del cambio.
   Uno no detecta los cambios en los que no está enfocado
   Si estoy enfocado en cierta parte de la pantalla, hay una parte que está cambiando y no la voy a ver si me pierdo esta información.
   Enfocar la atención del usuario en un solo lado, que es donde estoy dando la instrucción, procura que las instrucciones que se den con una sola forma. \\\hline

   Podrías simplificar mucho más el primer acercamiento. Tutoriales interactivos de lenguajes de programación: Esto es un if, lo que está dentro, así conectas lo que hiciste en el primer, paso con un conocimiento nuevo en el segundo paso \\\hline

   Algo importante en UX/UI es nunca sorprender al usuario. Mientras menos tenga que aprender, mejor. Lo ideal es usar estándares y colgarde de ellos. Piensa en la Ley de Jacob: Los usuarios gastan la mayor parte de su tiempo en otros sitios. Tu juego representa un porcentaje enano en la vida de los demás usuario.  \\

   \hline
   \end{tabular}
\end{table}

\restoregeometry

\chapter*{Anexo H: Tutoriales}

\label{AnexoHTutoriales}

\subsubsection{Primer Tutorial}

En el primer tutorial, se pretende familiarizar al jugador con la pantalla, demostrando que los planetas son navegables y que existen caminos que los conectan, permitiendo hacer clic entre ellos. En esta etapa, también se proporciona información sobre el diálogo y se presentan pistas visuales.


\begin{figure}[h]
	\centering
	\includegraphics[scale=0.3]{imagenes/FirstTutorial.png}
	\caption{Primer tutorial. Se le indica al jugador a través de un diálogo que debe visitar cada planeta.}
	\label{FirstTutorial}
\end{figure}



\subsubsection{Segundo tutorial}

En el segundo tutorial, se presenta la mecánica de instrucciones e introduce el ciclo for y el if en las instrucciones, elementos que se utilizarán más adelante.

Al inicio del segundo tutorial, se le indica al jugador que debe seguir las instrucciones del algoritmo debido a que la exploración automatizada de los pandas rojos es costosa y el combustible se está agotando. Las primeras dos instrucciones solicitan al jugador que presione la tecla espacio para avanzar a la siguiente instrucción (ver figura \ref{SecondTutorial}).


\begin{figure}[h]
	\centering
	\includegraphics[scale=0.2]{imagenes/SecondTutorial.png}
	\caption{Segundo tutorial. Se le muestra un diálogo de entrada al jugador introduciéndole el nuevo concepto. Este es el primer momento en que se le muestran las instrucciones al jugador.}
	\label{SecondTutorial}
\end{figure}

Durante este segundo tutorial, se introduce la lógica de responder sí o no cuando aparece una instrucción con un if. El jugador debe dar la respuesta correcta para avanzar a la siguiente instrucción. Si responde incorrectamente, perderá y deberá reiniciar el nivel. En la figura \ref{SecondTutorialShowingIf}, se muestra la ventana que aparece al jugador al llegar a una instrucción con un if.

\begin{figure}[h]
	\centering
	\includegraphics[scale=0.3]{imagenes/SecondTutorialShowingIf.png}
	\caption{Segundo tutorial. Ventana de if que le aparece al jugador al llegar a una instruccion con un if. El jugador debe responder sí o no correctamente para avanzar}
	\label{SecondTutorialShowingIf}
\end{figure}

\subsubsection{Tercer tutorial}

El tercer tutorial introduce al jugador el concepto de Pilas (Stacks) y Colas (Queues) como estructuras de datos para almacenar nodos y luego obtenerlos en órdenes distintos. Además, se le introduce al jugador sobre las variables creadas y la variable seleccionada, la cual se usa para señalar a qué objeto se le agregará el nodo que el usuario presione.

\begin{figure}[h]
	\centering
	\includegraphics[scale=0.3]{imagenes/ThirdTutorialFirstDialogue.png}
	\caption{Tercer tutorial. Se le explica al usuario al inicio qué son las colas y stacks y cómo funcionan y en qué difieren.}
	\label{ThirdTutorialFirstDialogue}
\end{figure}


\begin{figure}[h]
	\centering
	\includegraphics[scale=0.3]{imagenes/ThirdTutorialStackExplanation.png}
	\caption{Tercer tutorial. Mostrando al usuario cómo funciona un stack a través de animaciones.}
	\label{ThirdTutorialStackExplanation}
\end{figure}


El objetivo de las instrucciones en este tutorial es que el jugador aprenda la mecánica para agregar nodos a un stack o a una pila. Además, se refuerza la mecánica de las instrucciones y ejecutar cada acción paso por paso.


\begin{figure}[h]
	\centering
	\includegraphics[scale=0.3]{imagenes/ThirdTutorialAddingNode.png}
	\caption{Tercer tutorial. El jugador debe presionar R sobre el planeta 2 para avanzar a la siguiente instrucción.}
	\label{ThirdTutorialAddingNodesToStacks}
\end{figure}


% \end{appendices}
\end{document}
