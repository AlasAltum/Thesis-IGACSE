\chapter{Diseño de la Solución}

% Contenidos basados en https://repositorio.uchile.cl/bitstream/handle/2250/191381/TablaConten.pdf?sequence=2&isAllowed=y
La aplicación creada, llamada IGACSE (Interactive Graph Algorithms for Computer Science Education) de ahora en adelante, es 
propuesta y diseñada en base a diversos requerimientos y necesidades de los estudiantes de computación.

\section{Referentes}

Se tiene como supuesto de que muchas veces los estudiantes creen entender cómo funciona cierto código o algoritmo, pero no lo 
aplican paso a paso con papel y lápiz o en su imaginación propia. Cuando se asigna como ejercicio realizar el procedimiento 
siguiendo cada instrucción rigurosamente, los estudiantes no lo realizan, o lo hacen de forma incorrecta sin darse cuenta.

El videojuego logrado debe evitar la mecanización por parte del estudiante, obligándolo a revisar exhaustivamente cada paso
relacionado con el algoritmo, de manera que no sé por sentado que se entiende el código o de que se comprendió a cabalidad.

Por otra parte, para que enseñar conceptos relacionados con programación, lo ideal es buscar usuarios que sepan cómo funciona, pero
que no tengan tanta experiencia y que no necesariamente visualicen cómo funciona instrucción por instrucción. Por lo mismo, se 
propone basarse en un modelo similar a los debuggers, donde se puede ver el estado de las variables en cada paso, así como la 
instrucción por ejecutarse y el resultado de la misma.

\section{Descripción de la aplicación al momento de la realización de los experimentos}

El videojuego se separa en distintos niveles. Consta de un menú principal, tutoriales y niveles jugables. 
En el menú principal se puede seleccionar el nivel a jugar o un modo historia. En el modo historia, se juegan todos los 
niveles en orden, y se desbloquean a medida que se avanza.

En los tutoriales se enseñan los conceptos básicos de grafos, pero sin indicarle al usuario explícitamente qué es un grafo. 
Además, se enseñan conceptos de jugabilidad, cómo explorar o seleccionar un nodo, cómo navegar en el código ejecutando instrucciones, 
y cómo contestar preguntas del tipo Sí/No cuando el código tiene una instrucción que incluya un if.

Finalmente, se presentan los niveles jugables, que corresponden a los dos algoritmos que se buscan enseñar: BFS (Breadth First Search) y 
DFS (Depth First Search). En estos niveles no se presenta historia y hay menos ayuda para el usuario.

% Mostrar aquí muchas imágenes del resultado por partes.
% Mostrar un diagrama de flujo dentro del juego. Menu => (Tutorial o Niveles). Tutorial => Niveles. Niveles => Créditos.
% Mostrar cómo funcionan los tutoriales, cada uno. Dar muestras de la historia o narrativa presentada.
\subsection{Diagrama de flujo de juego}

\subsection{Menú principal}

\subsection{Tutoriales}

\subsection{Niveles jugables}

\subsection{Narrativa}


% utiliza la analogía de planetas y rutas espaciales. Esta decisión se basó en que los videojuegos suelen poseer una narrativa, 
% la cual lo hace más atractivo. Además, una narrativa ayuda a aterrizar un concepto abstracto como pueden ser los grafos.

\section{Requisitos de usuario}
\section{Requisitos de software}
\section{Requisitos no funcionales}
\section{Proceso de diseño}
\subsection{Diseño de la interfaz de usuario}
\subsection{Diseño de mecánicas de juego}


\section{Arquitectura de software}

\section{Implementación de la solución}
% Mencionar que se usó Godot, GDScript, bla bla bla


